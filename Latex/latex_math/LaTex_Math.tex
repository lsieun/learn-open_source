\documentclass[UTF8]{ctexart}
\author{HelloWorld}
\title{常用数学符号的 LaTeX 表示方法}
\begin{document}
\maketitle

\section{特殊符号}

\subsection{单个字符}

在LaTex中,大部分键盘字符都可以直接输入,但是“\#,\$,\%,\{,\},\~{},$\backslash$,\^{},”在Tex中有特殊的用途,排版时需要在字符前面加入强制转换。

这些特殊字符应该按如下对应输入:

\#---------$\backslash$\#

\$---------$\backslash$\$

\%---------$\backslash$\%

\&---------$\backslash$\&

\{---------$\backslash$\{

\}---------$\backslash$\}

$\backslash$---------\$$\backslash$backslash\$

\^{}---------$\backslash$\^{}\{\}

别处又找来这些:

换行---------$\backslash$newline

\textbackslash---------$\backslash$textbackslash

\textcopyright---------$\backslash$textcopyright

\textregistered---------$\backslash$textregistered

$^\circ$C---------\$\^{}$\backslash$circ\$C

\pounds---------$\backslash$pounds 

\dots---------$\backslash$dots

\today---------$\backslash$today  

\LaTeX{}---------$\backslash$LaTeX\{\}

\TeX{}---------$\backslash$TeX\{\} 

\LaTeXe{}---------$\backslash$LaTeXe\{\}

没有空格 ab---------ab

小空格	a\,b---------a$\backslash$,b

大空格	a\ b---------a$\backslash$ b

quad空格 a \quad b---------a $\backslash$quad b

两个quad空格	a \qquad b---------a $\backslash$qquad b


\subsection{空格\&换行}

空格:西文单词用空格隔开,TEX中的连续多个空格在编译排版时被看做一个空格。

换行:连续两个回车被看做是段落结束,可以使用$\backslash$$\backslash$强制换行。

\subsection{引号}

左单引号用键盘左上角的倒引号"`";

右单引号用键盘enter键旁边的单引号"'";

左双引号是连用两个倒引号"``";

右单引号连用两个单引号"''";

当单引号与双引号相邻时,在两者中间插入一个”$\backslash$,“。

\subsection{连字号、破折号}

一个连字号"-"表示一个连字号“-”

连续用两个连字号"-"表示数字范围符号“--”

连续用三个连字号"-"表示西文破折号“---”

(数学中的减号或者负号,需要输入\$-\$)

\subsection{连体号}

在Latex中,把ff这样的字母组合,当成一个连体号排版出来的,而不是分开来的。如要将两者分开,应在两者中间加入“$\backslash$/”,即“f$\backslash$/f”,效果如“f\/f”。

\subsection{句号后面的空白}

句号圆点:(1)可以表示句子结束,(2)也可以表示缩写。

在Latex中,小写字母后面圆点表示句子结束,但是为了表示缩写,需在圆点后倒斜线加上空格,即“$\backslash$ ”,表示缩写后面可以分行;


在Latex中,大写字母后的圆点看做是一个缩写而不是句号。有时大写字母后面圆点表示成句号,可以在圆点前加上$\backslash$@,即“$\backslash$@.”



\section{数学符号}

\subsection{数学模式重音符}

$\bar{a}$ \quad $\backslash$bar\{a\} \qquad
$\acute{a}$ \quad $\backslash$acute\{a\} \qquad
$\breve{a}$ \quad $\backslash$breve\{a\} \qquad
$\grave{a}$ \quad $\backslash$grave\{a\} \qquad

$\dot{a}$ \quad $\backslash$dot\{a\} \qquad
$\ddot{a}$ \quad $\backslash$ddot\{a\} \qquad
$\hat{a}$ \quad $\backslash$hat\{a\} \qquad
$\widehat{a}$ \quad $\backslash$widehat\{a\} \qquad

$\check{a}$ \quad $\backslash$check\{a\} \qquad
$\vec{a}$ \quad $\backslash$vec\{a\} \qquad
$\tilde{a}$ \quad $\backslash$tilde\{a\} \qquad
$\widetilde{a}$ \quad $\backslash$widetilde\{a\} \qquad

\subsection{小写希腊字母}

\subsection{大写希腊字母}

1、指数和下标可以用\^{}和\_后加相应字符来实现。比如:







\end{document}